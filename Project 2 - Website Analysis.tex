\documentclass{article}

\usepackage{hyperref}
\usepackage{array}
\usepackage[utf8]{inputenc}
\usepackage{chngcntr}
\counterwithin*{subsection}{section}
\renewcommand{\thesubsection}{\thesection.\alph{subsection}}

\title{Project 2 - Website Analysis}

\author{Purple Rabbits:\\Natalie Schneider and Thomas Merl}
\date{September 2 2021}

\begin{document}

\maketitle

\section{Background}

Natalie has had some professional experience creating offline versions of websites for educational resources, which were usually intended for distribution on flash drives with the educational material. These were generally intended to replace DVDs by being both higher quality and more accessible. The idea of bringing this online is to provide additional flexibility to distribution by allowing a link to be additionally provided to access the material.\vspace{\baselineskip}

Tommy has yet to have a pleasant experience with supplemental educational sites currently in use. MyMathLab, Person Online, etc. almost seem to degrade the education process more than they help.

\section{Website Analysis}
\subsection{Relevant Websites}
    
\textbf{Udemy} (\url{https://www.udemy.com/})

Educators list courses for sale at a set price and Udemy aggregates the courses. Each course has a review system based on a 5 star rating. Courses are divided up into categories and are then further divided into topics within those categories. Udemy has often sales to make courses look at a better value than they are.

Each course has a playlist, which is then separated into sections. Within these sections video lectures, quizzes, coding exercises, practice tests, and assignments can be added. There is a minimum amount of video hours required for the course.

To the user it appears as a constant flow of content, and each section is marked as completed after it is viewed. The video player has a 2x speed function, a captions function, and allows for taking notes at specific timecodes in the video. There are also ways to ask a forum of people who are in the course questions about the course. \vspace{\baselineskip}

\textbf{Skill Share} (\url{https://www.skillshare.com/})
Skill Share has a similar setup. Anyone may list a course on the platform
however they may be quick to notice the lack of rich content elements they can
add to there course. Without quizzes, code environments, etc., instructors are
left with building their course through video alone.

That being said, the ui is fluid and easy to use. The video player is fully
featured, with all the fixings found in Udamy. The largest point of friction is
that you are required to give a credit card, even if you use the free
plan.

The courses seem to be regularly updated with content, and are all at your own
pace. Instructors make money by the number of minutes watched from their
courses.

\textbf{EDx} (\url{https://www.edx.org/})
Edx provides a more traditional educational environment not too dissimilar to a 100%
online collage. Most courses are in fact provided by collages like MIT and NYU.
In some, one can even earn credit towards the linked institution, even leading
to an accredited degree. While there are at your own pace courses,
most, especial ones with linked institutions, follow an IRL schedule.

To support this more formal approach the site behaves more like a learning
management system akin to Canvas; Just one where you can pick your courses.
There is a quiz, grade, and work submission system. Work is often graded by a
teacher or TA with human feedback.

\textbf{Ted Ed} (\url{https://ed.ted.com/})
Ted-Ed is somewhat different than the others as it is a secondary project to
the larger TED Talk program. The site allows users to curate playlists of TED
talks / animations to be shared as courses. Users can also incorporate YouTube
videos into their playlists. These playlist are then publicly listed and made
easy shareable to students.

The site is very smooth to use and completely free. The TED library is massive,
however content is more surface level than that provided by other sites.

\textbf{Coursera} (\url{https://www.coursera.org/})
Last but not least, Coursera is one of the original sites of this style. With
this they provide an environment closest to Edx. Courses are built around short
video chunks with test checkpoints spaced throughout.

The ui is much less smooth than the other contenders; One has to weave through
a maze of drop downs to even browse available courses. What they do well though
is online discussion boards where classmates can chat about the course content.
From our investigation one need to contact Coursera to produce content for the
platfourm.


\subsection{Functions to Implement}



\subsection{Comparison Table}
\begin{tabular}{ | m{10em} | m{2cm}| m{1cm} | m{1.5cm}| m{1cm} | m{1cm}| m{1.5cm} | } 
  \hline
   & My proposed system & Udemy & Skill Share & EDx & Ted Ed & Coursera\\ 
  \hline
  Individual sale of courses & V & V & X & & & \\ 
  \hline
  User rating system &  & V & & & & \\ 
  \hline
  Categorization of courses on main page & V & V & & & & \\ 
  \hline
  Ability to set a sale/discout code & V & V & & & & \\ 
  \hline
  Playlist of Content & & V & & & & \\ 
  \hline
  Auto play between videos & & V & & & & \\ 
  \hline
  Supplementary content to videos & V & & & & & \\ 
  \hline
  Keeping notes in Videos & & V & & & & \\ 
  \hline
  Keep track of user's place in courses & & V & & & & \\ 
  \hline
  2X speed function & & V & & & & \\ 
  \hline
  Video Captions & & V & & & & \\ 
  \hline
  Note taking capabilities & & V & & & & \\ 
  \hline
  Supplementary forum to go with courses & & V & & & & \\ 
  \hline
  Function & & & & & & \\ 
  \hline
\end{tabular}
V: Able to perform the task; X: Unable to perform the task; O: Able to perform the task with poor interactive design
\end{document}
